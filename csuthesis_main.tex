\documentclass{CSUthesis}

%!TEX root = ../csuthesis_main.tex
% 文章信息
\titlecn{标题}
\titleen{English}

\priormajor{}
\minormajor{}
\interestmajor{}
\author{某人}
\supervisor{某导师}
\subsupervisor{}
\department{某学院}
\studentid{123456}
\thesisdate{year=1999,month=8}
\myclass{某专业某班}

\begin{document}
\renewcommand*{\baselinestretch}{1.5}   % 几倍行间距
\setlength{\baselineskip}{22.5pt}       % 基准行间距
%%%%%%%%%%%%%%%%%%%%%%%%%%%%%%%%%%%%%%%%%%%%%%%%%%
% 封面
% -----------------------------------------------%
\makecoverpage

%%%%%%%%%%%%%%%%%%%%%%%%%%%%%%%%%%%%%%%%%%%%%%%%%%
% 前置部分的页眉页脚设置
% -----------------------------------------------%
\newpage
% 正文和后置部分用阿拉伯数字编连续码,前置部分用罗马数字单独编连续码(封面除外)。
% 设置封面页后的页码
\pagenumbering{Roman} % 大写罗马字母
\setcounter{page}{1} % 从1开始编号页码
% 设置页眉和页脚
% 本科生从摘要开始就要有
% 设置页眉和页脚 %
\pagestyle{fancy}
\fancyhead[L]{\includegraphics[scale=0.13]{logo/logo.png}}
\fancyhf[RH]{\heiti \zihao{-5} {面向云原生边缘计算平台的云边调度算法}} % 设置所有(奇数和偶数)右侧页眉


%%%%%%%%%%%%%%%%%%%%%%%%%%%%%%%%%%%%%%%%%%%%%%%%%%
% 中文摘要
% -----------------------------------------------%
%!TEX root = ../csuthesis_main.tex
% 设置中文摘要
\keywordscn{关键词\quad 关键词\quad 关键词}
%\categorycn{TP391}
\begin{abstractcn}
  摘要。
\end{abstractcn}
\newpage

%%%%%%%%%%%%%%%%%%%%%%%%%%%%%%%%%%%%%%%%%%%%%%%%%%
% 英文摘要
% -----------------------------------------------%
\renewcommand*{\baselinestretch}{1.0} 
%!TEX root = ../csuthesis_main.tex
\keywordsen{Key\ \ Key\ \ Key}
\begin{abstracten}
  Abstract.
\end{abstracten}
\newpage

% 目录
% -------------------------------------------%

{

\renewcommand{\contentsname}{\hfill \heiti \zihao{3} 目\quad 录\hfill}
	\renewcommand*{\baselinestretch}{1.435}   % 行间距
    \tableofcontents
}
\newpage
% 去掉页眉章节序号后面的“.”
\renewcommand{\sectionmark}[1]{\markright{\thesection~ #1}}


\renewcommand{\headrulewidth}{1pt}

% 正文内容
% --------------------------------------------%
\setcounter{page}{1} % 重置页码编号
\pagenumbering{arabic} % 设置页码编号为阿拉伯数字

% LastPage是从lastpage宏包中引来的
\fancyfoot[C]{\songti 第 \thepage 页\quad 共 \pageref{LastPage} 页} % 所有(奇数和偶数)中间页脚


% 可以使用include命令导入tex文件,从而避免过多修改本文件。

% 论文正文是主体,主体部分应从另页右页开始,每一章应另起页。一般由序号标题、文字叙述、图、表格和公式等五个部分构成。

% 重新设置正文行间距,因为前置部分设置时候行间距被改过
\renewcommand*{\baselinestretch}{1.5}   % 几倍行间距


% 正文
%!TEX root = ../csuthesis_main.tex

%子章节为了便于查找和修改,建议通过input拆分文件

%!TEX root = ../../csuthesis_main.tex
\section{引言}
引言。

%!TEX root = ../../csuthesis_main.tex
\section{技术基础}
技术基础。

\section{研究内容及实验方案}
研究内容及实验方案。

\section{难点及初步解决办法}
难点及初步解决办法。

\section{预期成果及时间安排}
预期成果及时间安排。

\section{结论}
结论。


\newpage


%%%%%%%%%%%%%%%%%%%%%%%%%%%%%%%%%%%%%%%%%%%%%%%%%%
% 临时标签,用于编译时追踪正文末尾
%%%%%%%%%%%%%%%%%%%%%%%%%%%%%%%%%%%%%%%%%%%%%%%%%%

%%%%%%%%%%%%%%%%%%%%%%%%%%%%%%%%%%%%%%%%%%%%%%%%%%
% 后续内容,标题三号黑体居中,章节无编号
% --------------------------------------------%

% https://www.zhihu.com/question/29413517/answer/44358389 %
% 说明如下:
% secnumdepth 这个计数器是 LaTeX 标准文档类用来控制章节编号深度的。在 article 中,这个计数器的值默认是 3,对应的章节命令是 \subsubsection。也就是说,默认情况下,article 将会对 \subsubsection 及其之上的所有章节标题进行编号,也就是 \part, \section, \subsection, \subsubsection。LaTeX 标准文档类中,最大的标题是 \part。它在 book 和 report 类中的层级是「-1」,在 article 类中的层级是「0」。这里,我们在调用 \appendix 的时候将计数器设置为 -2,因此所有的章节命令都不会编号了。不过,一般还是会保留 \part 的编号的。所以在实际使用中,将它设置为 0 就可以了。

% 在修改过程中请注意不要破环命令的完整性

\renewcommand\appendix{\setcounter{secnumdepth}{-2}}
\appendix

% 主文件有代码去掉页眉章节编号的“.”,但这会因为bug导致无编号章节显示一个错误编号,所以这里在无编号章节之前再次重定义sectionmark。
\renewcommand{\sectionmark}[1]{\markright{#1}}

% 没有致谢。

% section 标题从这里往后改为三号黑体居中
\titleformat{\section}{\centering \zihao{3}\heiti}{\thesection}{1em}{}

% \section{参考文献}
\addcontentsline{toc}{section}{参考文献}
\begin{thebibliography}{}
  \bibitem{ref1} 摘要。
\end{thebibliography}


\end{document}
